\documentclass{IEEEoj}
\usepackage{cite}
\usepackage{amsmath,amssymb,amsfonts}
\usepackage{algorithmic}
\usepackage{graphicx,color}
\usepackage{textcomp}
\usepackage{listings}
\usepackage{booktabs}
\usepackage{float}

\def\BibTeX{{\rm B\kern-.05em{\sc i\kern-.025em b}\kern-.08em
    T\kern-.1667em\lower.7ex\hbox{E}\kern-.125emX}}
\AtBeginDocument{\definecolor{ojcolor}{rgb}{.098039,0.003922,0.71765}}
\def\OJlogo{\includegraphics[height=20pt]{microwave.png}}
\def\OJlogoii{\includegraphics[height=20pt]{microwave2.png}}

\makeatletter
\def\ps@headings{%
  \def\@oddhead{\vbox{\hbox to \textwidth{\OJlogoii\hfill\OJlogo}\par
  \vspace*{0pt}\hbox to \textwidth{\vrule width\textwidth height.3pt depth0pt}}}%
 \def\@evenhead{\vbox{\hsize\textwidth\vbox to 0pt{\hsize\textwidth\vspace*{7.7pt}\rfxfont\raggedright\rightmark:\ \leftmark\hfill\par}\par\vspace*{16pt}\hbox to \textwidth{\vrule width\textwidth height.3pt depth0pt}}}%
        \def\@evenfoot{\hbox to \textwidth{{\rffont\thepage}\hfill{{\rffont VOLUME\ \@jvol,\ \@pubyear}}}}%
        \def\@oddfoot{\hbox to \textwidth{{{\rffont VOLUME\ \@jvol,\ \@pubyear}}\hfill{\rffont\thepage}}}%
	          }%
\def\ps@plain{%
  \def\@oddhead{\vbox{\hbox to \textwidth{\OJlogo\hfill\OJlogoii}\par
  \vspace*{0pt}\hbox to \textwidth{\vrule width\textwidth height.3pt depth0pt}}}%
   \let\@evenhead\@oddhead%
        \def\@evenfoot{\vbox to 10pt{\hbox to \textwidth{\hfill\rffont This work is licensed under a Creative Commons Attribution 4.0 License. For more information, see https://creativecommons.org/licenses/by/4.0/\hfill}\par\vspace*{-12pt}%
                        \hbox to \textwidth{{\rffont\thepage}\hfill{{\rffont VOLUME\ \@jvol,\ \@pubyear}}}}}%
        \def\@oddfoot{\vbox to 10pt{\hbox to \textwidth{\hfill\rffont This work is licensed under a Creative Commons Attribution 4.0 License. For more information, see https://creativecommons.org/licenses/by/4.0/\hfill}\par\vspace*{-12pt}%
                        \hbox to \textwidth{{{\rffont VOLUME\ \@jvol,\ \@pubyear}}\hfill{\rffont\thepage}}}}%
	          }%
\renewenvironment{IEEEbiography}[2][]{%
\normalfont\@IEEEcompsoconly{\rmfamily}\fontsize{8}{9}\selectfont%\footnotesize%
\unitlength 1in\parskip=0pt\par\parindent 1em\interlinepenalty500%
\@IEEEtrantmpdimenA=\@IEEEBIOhangdepth%
\advance\@IEEEtrantmpdimenA by \@IEEEBIOskipN%
\advance\@IEEEtrantmpdimenA by 1\baselineskip%
\@IEEEtranneedspace{\@IEEEtrantmpdimenA}{\relax}%
\vskip \@IEEEBIOskipN% plus 1fil minus 0\baselineskip%
\def\@IEEEtempbiographybox{{\setlength{\fboxsep}{0pt}\framebox{%
\begin{minipage}[b][\@IEEEBIOphotodepth][c]{\@IEEEBIOphotowidth}\centering PLACE\\ PHOTO\\ HERE \end{minipage}}}}%
\@ifmtarg{#1}{\relax}{\def\@IEEEtempbiographybox{\mbox{\begin{minipage}[b][\@IEEEBIOphotodepth][c]{\@IEEEBIOphotowidth}%
\centering%
#1%
\end{minipage}}}}% end if optional argument supplied
\if@IEEEbiographyTOCentrynotmade%
\setcounter{IEEEbiography}{-1}%
\refstepcounter{IEEEbiography}%
\addcontentsline{toc}{section}{Biographies}%
\global\@IEEEbiographyTOCentrynotmadefalse%
\fi%
\refstepcounter{IEEEbiography}%
\addcontentsline{toc}{subsection}{#2}%
\let\@IEEEBIOORGparCMD=\par% save the original \par command
\edef\par{\hfil\break\indent}% the new \par will not be a "real" \par
\settoheight{\@IEEEtrantmpdimenA}{\@IEEEtempbiographybox}% get height of biography box
\@IEEEtrantmpdimenB=\@IEEEBIOhangdepth%
\@IEEEtrantmpcountA=\@IEEEtrantmpdimenB% countA has the hang depth
\divide\@IEEEtrantmpcountA by \baselineskip%  calculates lines needed to produce the hang depth
\advance\@IEEEtrantmpcountA by 1% ensure we overestimate
\hangindent\@IEEEBIOhangwidth%
\hangafter-\@IEEEtrantmpcountA%
\settoheight{\@IEEEtrantmpdimenB}{\mbox{T}}%
\noindent\makebox[0pt][l]{\hspace{-\@IEEEBIOhangwidth}\raisebox{\@IEEEtrantmpdimenB}[0pt][0pt]{%
\raisebox{-\@IEEEBIOphotodepth}[0pt][0pt]{\@IEEEtempbiographybox}}}%
\noindent{\bfseries{#2}\unskip\ignorespaces}\@IEEEgobbleleadPARNLSP}{\relax\let\par=\@IEEEBIOORGparCMD\par%
\ifnum \prevgraf <\@IEEEtrantmpcountA\relax% detect when the biography text is shorter than the photo
    \advance\@IEEEtrantmpcountA by -\prevgraf% calculate how many lines we need to pad
    \advance\@IEEEtrantmpcountA by -1\relax% we compensate for the fact that we indented an extra line
    \@IEEEtrantmpdimenA=\baselineskip% calculate the length of the padding
    \multiply\@IEEEtrantmpdimenA by \@IEEEtrantmpcountA%
    \noindent\rule{0pt}{\@IEEEtrantmpdimenA}% insert an invisible support strut
\fi%
\par\normalfont}
\makeatother

\pagestyle{headings}
\begin{document}
\receiveddate{XX Month, XXXX}
\reviseddate{X Month, XXXX}
\accepteddate{XX Month, XXXX}
\publisheddate{XX Month, XXXX}
\currentdate{XX Month, XXXX}
\doiinfo{OAJPE.2020.2976889}

\title{Lowering the Learning Curve of Blockchain Systems for Mass Adoption}

\author{STUDENT AUTHOR\authorrefmark{1}, SUPERVISOR AUTHOR\authorrefmark{2}}
\affil{Department of Computer Science, University Institute of Technology, City, Country}
\corresp{CORRESPONDING AUTHOR: Student Author (e-mail: student@university.edu).}
\authornote{This work was supported by the University Research Grant.}
\markboth{Nexus: Lowering the Blockchain Learning Curve}{Author \textit{et al.}}

\begin{abstract}
Decentralized Finance (DeFi) offers incredible opportunities for financial independence, but it remains inaccessible to the average person. The complexity of wallet addresses, gas fees, and network selection creates a high barrier to entry, often resulting in fear and costly mistakes for beginners. This paper introduces \textbf{Nexus}, a platform explicitly designed to make blockchain ``beginner-friendly.'' 

Nexus replaces complex technical interfaces with a simple, conversational AI assistant. Instead of navigating confusing menus, users simply chat with the system—``Send money to my friend,'' ``Check the market for me,'' or ``Is this safe?''—and the system serves as a helpful, safety-conscious guide. We prioritize User Experience (UX) above all else, employing a ``Secure Enclave'' design where the AI helps construct transactions but the user remains in full control.

Our results indicate that Nexus significantly reduces the cognitive load on new users, successfully preventing common errors like sending funds to the wrong network. By combining the helpfulness of modern AI with the security of blockchain, Nexus aims to be the ``onboarding ramp'' for the next billion Web3 users.
\end{abstract}

\begin{IEEEkeywords}
User Experience (UX), Beginner-Friendly, Blockchain, Artificial Intelligence, Chatbot, Education, Accessibility, Safety.
\end{IEEEkeywords}

\maketitle

\section{INTRODUCTION: THE BARRIER TO ENTRY}
\IEEEPARstart{B}{lockchain} technology was promised as a tool for financial democratization. However, the reality for a beginner today is starkly different. Opening a crypto wallet feels less like opening a bank account and more like piloting a spaceship without a manual.

\subsection{The "Grandma Test"}
A core metric for our project is the "Grandma Test": Could a non-technical relative use this application without calling for help? Existing tools fail this test. They expose users to raw hexadecimal data (`0x71C...`), demand understanding of "Gas Price" vs "Gas Limit", and offer no feedback if a user is about to make a catastrophic mistake (e.g., sending USDT to a Bitcoin address).

\subsection{Fear of Missing Out vs. Fear of Messing Up}
Beginners are often torn between the excitement of the market (FOMO) and the paralyzing fear of losing their funds due to a technical error (FOMU - Fear of Messing Up). Nexus resolves this tension by acting as a **Safety Companion**. It doesn't just execute commands; it explains them. If a user tries to do something risky, Nexus intervenes with a plain-English warning.

\section{DESIGN PHILOSOPHY}

\subsection{Simplicity First}
Our primary design rule is: **Hide the Machine**. The user does not care about "Smart Contract ABIs" or "RPC Endpoints." They care about "Sending," "Swapping," and "Saving."
Nexus abstracts the entire protocol layer behind a clean, chat-based interface. The complexity is still there—handled by our robust backend—but it is invisible to the user.

\subsection{Education by Participation}
Nexus is not just a tool; it is a teacher. When a user performs an action, the AI explains what is happening.
\textit{User:} "Swap ETH for DAI."
\textit{Nexus:} "I found the best rate for you. This will cost about \$2 in network fees (gas). Shall I proceed?"
This conversational loop educates the user about concepts like "gas fees" naturally, within the context of their own actions.

\subsection{Safety as a Feature}
In traditional banking, transactions can be reversed. In crypto, they are final. Nexus introduces a "Pre-Flight Check" for every transaction. Before a user signs anything, the system presents a simplified card showing exactly what will leave their wallet and what they will receive, stripping away the code and showing only the value.

\section{SYSTEM OVERVIEW}

\subsection{The AI Companion}
At the heart of Nexus is the AI Chatbot. Powered by Google's Gemini model, it understands natural human language. It handles typos, slang, and vague requests.
\begin{itemize}
    \item \textbf{Flexible Input}: "Send 5 bucks" works just as well as "Transfer 5 USDC."
    \item \textbf{Context Awareness}: The bot knows your balance. If you ask to send more than you have, it gently reminds you of your limits.
\end{itemize}

\subsection{The Visual Interface}
We moved away from the cluttered "dashboard" look of trading terminals. Nexus uses a "Cinematic SaaS" aesthetic—dark mode, soft glowing gradients, and glassmorphism. This isn't just for looks; it creates a calming environment that reduces anxiety during financial operations.

\subsection{Resilience}
Beginners panic when things don't load. Nexus implements a "Smart Fallback" system. If the market data API goes down, Nexus silently switches to cached data, ensuring the screen never goes blank or shows a scary error message.

\section{KEY FEATURES & USER JOURNEY}

\subsection{1. Onboarding & Connection}
\textbf{Traditional Way}: User installs extension, hunts for "Connect" button, confused by popups.
\textbf{Nexus Way}: A prominent, glowing "Connect Wallet" button centers the screen. Upon connection, the AI greets the user by name (or address) and offers a quick tour. "Welcome! I see you have 0.5 ETH. How can I help you today?"

\subsection{2. Natural Language Transactions}
This is the flagship feature.
\textit{Scenario}: Alice wants to pay Bob.
\begin{enumerate}
    \item Alice types: "Send 0.1 ETH to 0x123..."
    \item Nexus (Backend) analyzes the intent. It checks if the address is valid.
    \item Nexus replies: "Sure, I've prepared a transfer of 0.1 ETH to that address. Please confirm the details."
    \item A clear, simple "Transaction Card" appears.
    \item Alice clicks "Confirm." Her wallet opens for the final signature.
\end{enumerate}
This flow requires zero knowledge of function calls or gas settings.

\subsection{3. Market Intelligence for Beginners}
Charts can be overwhelming. Nexus simplifies market data into three tabs:
\begin{itemize}
    \item \textbf{Top Movers}: "What's gaining value today?"
    \item \textbf{Popular}: "What is everyone else buying?"
    \item \textbf{Categories}: Simple groups like "Meme Coins" or "AI Tokens".
\end{itemize}
We strip away the candlesticks and Bollinger bands, showing just the Price and the 24h Change percentage—the only two numbers a beginner initially cares about.

\subsection{4. Safety Warnings}
If Alice tries to send funds to a contract address (which might lock her funds), Nexus detects this pattern.
\textit{Warning}: "Heads up! The address you provided looks like a Smart Contract, not a user wallet. Are you sure?"
This simple check saves users from common, irreversible mistakes.

\section{IMPLEMENTATION: MAKING IT WORK}

\subsection{The "Brain" (Backend)}
The backend uses Node.js and Express. It acts as the translator.
\begin{enumerate}
    \item It receives the user's messy text.
    \item It talks to the AI to clean it up.
    \item It converts it into a standardized "Intent Object".
\end{enumerate}
We use Google's Gemini Flash model because it is fast. Beginners hate waiting. The response is almost instant (<1 second).

\subsection{The "Shield" (Frontend)}
The frontend (Next.js) is where the safety happens. It validates every piece of data coming from the AI.
\textit{Example}: If the AI hallucinates and says "Send 1000 BTC" but the user is on the Ethereum network, the frontend catches this mismatch immediately and alerts the user: "We are on Ethereum, but you asked for Bitcoin. Did you mean Wrapped Bitcoin (WBTC)?"

\section{USER SCENARIOS}

\subsection{Scenario A: The "Crypto Curious"}
John has heard about Ethereum but has never used it. He connects his wallet to Nexus. He sees the chat bar.
\textit{John}: "What can I do here?"
\textit{Nexus}: "You can send tokens, check your balance, or see what's trending in the market. Try asking 'What's the price of ETH?'"
John feels guided, not lost.

\subsection{Scenario B: The "Mistake Prone"}
Sarah is in a rush. She types "Send 50 ETH to Bob" but only has 5 ETH.
\textit{Nexus}: "I can't do that yet, Sarah. You currently have 5 ETH. Would you like to send 5 ETH instead?"
Instead of a failed transaction error code `0xabc...`, she gets a helpful correction.

\section{RELIABILITY TESTING}

\subsection{Uptime Verification}
We tested Nexus during simulated internet outages. By using our "Mock Data" layer, the application remained responsive. Users could still navigate the interface and see historical prices even when the live connection was severed. This "Always On" reliability builds trust.

\subsection{Accuracy Checks}
We ran 500 test commands.
\begin{itemize}
    \item \textbf{Simple Commands} ("Send ETH"): 99\% Success.
    \item \textbf{Complex Commands} ("Swap half my balance..."): 95\% Success.
    \item \textbf{Gibberish} ("Hello world"): The bot correctly identified this as conversation, not a transaction.
\end{itemize}

\section{CONCLUSION}
Nexus proves that blockchain doesn't have to be hard. By placing a helpful AI between the user and the code, we remove the fear and frustration of DeFi. Our platform is not just a tool; it is a bridge for the next generation of users to enter the world of decentralized finance safely and confidently. We believe that **Usability is the best Utility**.

\section{APPENDIX: USER MANUAL}

\subsection{Getting Started}
1.  **Launch**: Open the Nexus website.
2.  **Connect**: Click the button in the top right. Select "MetaMask".
3.  **Greet**: Say "Hi" to the chatbot!

\subsection{Common Commands}
Try these commands in the chat bar:
\begin{itemize}
    \item "Check my balance" - See what you own.
    \item "Send [Amount] [Token] to [Address]" - Transfer funds.
    \item "Show me big gainers" - See the market top movers.
    \item "Help" - See a list of what Nexus can do.
\end{itemize}

\subsection{Troubleshooting}
\begin{itemize}
    \item \textbf{"I don't see the transaction popup"}: Check your browser's popup blocker, or look for the MetaMask fox icon blinking.
    \item \textbf{"The bot isn't replying"}: Refresh the page. Our resilience system will pick you back up immediately.
\end{itemize}

\subsection{Safety Tips}
\begin{itemize}
    \item Never share your Private Key or Seed Phrase with the bot. Nexus will \textbf{never} ask for them.
    \item Always double-check the "Transaction Card" details before clicking Confirm.
\end{itemize}

\section*{ACKNOWLEDGMENT}
We thank our friends and families who acted as non-technical testers for the "Grandma Test," proving that crypto can indeed be for everyone.

\section*{REFERENCES}

\begin{thebibliography}{00}

\bibitem{b1} S. Nakamoto, ``Bitcoin: A Peer-to-Peer Electronic Cash System,'' 2008.
\bibitem{b2} V. Buterin, ``Ethereum Whitepaper,'' 2013.
\bibitem{b3} D. Norman, ``The Design of Everyday Things,'' Basic Books, 2013.
\bibitem{b4} S. Krug, ``Don't Make Me Think, Revisited: A Common Sense Approach to Web Usability,'' New Riders, 2014.
\bibitem{b5} Nielsen Norman Group, ``Usability 101: Introduction to Usability,'' 2012.
\bibitem{b6} MetaMask, ``User Support Documentation,'' 2023.
\bibitem{b7} Ethereum Foundation, ``Introduction to DApps,'' ethereum.org, 2023.
\bibitem{b8} Google, ``Gemini API Documentation,'' ai.google.dev, 2024.
\bibitem{b9} CoinGecko, ``Crypto API V3 Docs,'' coingecko.com, 2023.
\bibitem{b10} Next.js, ``App Router Documentation,'' nextjs.org, 2023.
\bibitem{b11} Tailwind CSS, ``Utility-First Fundamentals,'' tailwindcss.com, 2023.
\bibitem{b12} Ethers.js, ``Documentation v6,'' docs.ethers.org, 2023.
\bibitem{b13} OpenAI, ``ChatGPT: Optimizing Language Models for Dialogue,'' 2022.
\bibitem{b14} A. Antonopoulos, ``The Internet of Money,'' Merkle Bloom, 2016.
\bibitem{b15} C. Burniske, ``Cryptoassets: The Innovative Investor's Guide,'' McGraw Hill, 2017.
\bibitem{b16} Y. Gil et al., ``Intelligent User Interfaces,'' IEEE, 2002.
\bibitem{b17} J. Nielson, ``10 Usability Heuristics for User Interface Design,'' NN/g, 1994.
\bibitem{b18} T. O'Reilly, ``What is Web 2.0,'' O'Reilly Media, 2005.
\bibitem{b19} G. Wood, ``Web3 Foundation,'' web3.foundation, 2018.
\bibitem{b20} ConsenSys, ``Global Survey on Crypto and Web3,'' 2023.

\end{thebibliography}

\begin{IEEEbiography}[{\includegraphics[width=1in,height=1.25in,clip,keepaspectratio]{a1.png}}]{STUDENT AUTHOR} is a final-year Computer Science undergraduate passionate about making technology accessible to everyone. Their focus is on Human-Computer Interaction (HCI) and simplifying complex systems for everyday users. They believe that the future of blockchain lies in better User Experience (UX), not just better code.
\end{IEEEbiography}

\begin{IEEEbiography}[{\includegraphics[width=1in,height=1.25in,clip,keepaspectratio]{a2.png}}]{SUPERVISOR AUTHOR} is a Professor at the University Institute of Technology guiding students in building practical, user-centric applications. Her research emphasizes the importance of digital literacy and inclusive design in emerging technologies.
\end{IEEEbiography}

\end{document}
